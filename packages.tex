\usepackage{lastpage}
\usepackage{tabularx}
\usepackage{cprotect}
%Seitenränder definieren
\usepackage[right=2.8cm,left=2.5cm, bottom=3.9cm, top=4.1cm, footskip=2.1cm, headsep=2.0cm]{geometry}
\usepackage[utf8]{inputenc}
\usepackage[T1]{fontenc}
\usepackage{palatino}
\linespread{1.25}
\usepackage{microtype}
\usepackage[english]{babel}
%More citations
\usepackage{natbib}
%better urls mit \url{}
\usepackage[hyphens]{url}
\usepackage{lastpage}
%code
\usepackage{listings}

%more colors
\usepackage{soul}

%Force figure to be placed “HERE”
\usepackage{float} 

%folgende Zeilen sind für Kapitelüberschriften
\usepackage[rigidchapters]{titlesec}
\usepackage{blindtext}
\titleformat{\chapter}
{\normalfont\LARGE}
{\makebox[3pc][l]{\LARGE\thechapter\hfil\rule[-6pt]{0.5pt}{2pc}}}
{0pt}
{\LARGE}
\titlespacing*{\chapter}{0pt}{0pt}{82pt}

%Csv -> Latex
\usepackage{csvsimple}

%Für Graphiken 
\usepackage{tikz}
\usetikzlibrary{plotmarks}
\usetikzlibrary{positioning,shapes,shadows,arrows}
\usepackage{graphicx}

%Paket gibt einige Optionen mehr bei Tabellen (wird eher nicht verwendet)
\usepackage{array}

%\usepackage{stdpage}
%test wegen anzahl zeilen pro seite
%Paket für Zeilenabstände
\usepackage{setspace}
%\onehalfspacing
\usepackage{multirow}
%Paket gibt mehr Kontrolle über die Captions (Bildunterschriften) bei Abbildungen
\usepackage[labelfont=bf,format=hang,font=footnotesize,justification=raggedright,singlelinecheck=false]{caption}

%Helvetia (Arial) Verwenden WICHTIG: Beide folgenden Zeilen kopieren!
%\usepackage[scaled]{helvet} %
%\renewcommand*\familydefault{\sfdefault} %%

% Abschalten des Einrückens bei neuen Absätzen (manuell, nach Tabellen, Abbildungen, etc.)
\setlength{\parindent}{0pt}
\hyphenation{}

% New Page before section
\newcommand{\sectionbreak}{\clearpage}

\usepackage{color}

\definecolor{color0}{rgb}{0,0,0}% black
\definecolor{color1}{rgb}{0.22,0.45,0.70}% light blue
\definecolor{color2}{rgb}{0.45,0.45,0.45}% dark grey
\definecolor{mygreen}{rgb}{0,0.6,0}
\definecolor{mygray}{rgb}{0.5,0.5,0.5}
\definecolor{myblue}{rgb}{0.0, 0.53, 0.74}
\definecolor{codebackground}{rgb}{0.8, 0.8, 0.8}
\definecolor{codechanged}{rgb}{0.8, 0.0, 0.0}

\lstset{ %
  backgroundcolor=\color{codebackground},   % choose the background color; you must add \usepackage{color} or \usepackage{xcolor}
  basicstyle=\footnotesize,        % the size of the fonts that are used for the code
  breakatwhitespace=false,         % sets if automatic breaks should only happen at whitespace
  breaklines=true,                 % sets automatic line breaking
  captionpos=b,                    % sets the caption-position to bottom
  commentstyle=\color{mygreen},    % comment style
  deletekeywords={...},            % if you want to delete keywords from the given language
  escapeinside={\%*}{*)},          % if you want to add LaTeX within your code
  extendedchars=true,              % lets you use non-ASCII characters; for 8-bits encodings only, does not work with UTF-8
% frame=single,	                   % adds a frame around the code
  keepspaces=true,                 % keeps spaces in text, useful for keeping indentation of code (possibly needs columns=flexible)
% keywordstyle=\color{blue},       % keyword style
% language=Octave,                 % the language of the code
% otherkeywords={*,...},           % if you want to add more keywords to the set
  numbers=left,                    % where to put the line-numbers; possible values are (none, left, right)
  numbersep=5pt,                   % how far the line-numbers are from the code
  numberstyle=\tiny\color{mygray}, % the style that is used for the line-numbers
  rulecolor=\color{black},         % if not set, the frame-color may be changed on line-breaks within not-black text (e.g. comments (green here))
  showspaces=false,                % show spaces everywhere adding particular underscores; it overrides 'showstringspaces'
  showstringspaces=false,          % underline spaces within strings only
  showtabs=false,                  % show tabs within strings adding particular underscores
  stepnumber=2,                    % the step between two line-numbers. If it's 1, each line will be numbered
% stringstyle=\color{mymauve},     % string literal style
  tabsize=2,	                   % sets default tabsize to 2 spaces
% title=\lstname,                   % show the filename of files included with \lstinputlisting; also try caption instead of title
  moredelim=**[is][\color{codechanged}]{**@}{@**}, %Rot für Änderungen
  moredelim=**[is][\color{myblue}]{***@}{@***}, %einfaches blau
  moredelim=**[is][\color{mygreen}]{*@}{@*},%einfaches Grün
}

%Aktives Inhaltsverzeichnis und links
\usepackage{hyperref}
\hypersetup{
    colorlinks,
    citecolor=black,
    filecolor=black,
    linkcolor=black,
    urlcolor=black
}

\usepackage{fancyhdr}
\pagestyle{fancy}
\fancypagestyle{plain}{}
\fancyhf{}
\fancyfoot{} % clear all footer fields

\fancyhead[R]{\small{\leftmark}}
\fancyhead[L]{\small{Offensive Security - Penetration Test Report}}
\renewcommand{\sectionmark}[1]{\markboth{#1}{}}

\renewcommand{\footrulewidth}{0.1pt} % Create a rule above the page number
\fancyfoot[R]{\textcolor{color1}\thepage  \textcolor{color2}{/\pageref{LastPage}}}